\documentclass[12pt]{article}
\usepackage{color}
\usepackage{tabularx}

\begin{document}
\begin{titlepage}
\title{\textbf{CPSC 504 Project\\\vspace{3 cm}\Huge{An investigation into an application to manage data transformations and updates}\vspace{2 cm}}}
\date{\vspace{2 cm} April 14, 2015}

\author{
 \makebox[1.0\linewidth]{Laura Cang}\\cang@cs.ubc.ca\\
  \and \makebox[1.0\linewidth]{Kailang Jiang}\\jiangkl@cs.ubc.ca\\
  \and \makebox[1.0\linewidth]{Jessica Wong}\\jhmwong@cs.ubc.ca\\ 
}
\maketitle
\thispagestyle{empty}
\end{titlepage}

\newpage
\section{Introduction}
\label{sec:introduction}
This project is built on top of the theoretical framework done in the M.Sc thesis of Arni Thrastarson \cite{arniThesis} and focuses on the first of the four scenarios detailed in the work. The first scenario investigates how changes made by a data application can be captured and propagated to a remote data source. The main workflow likely involves a (possibly third party) application pulling the data from a remote data source to use locally. The application creates a local copy of the data along with a schema mapping to identify the source table(s) of the data attributes. A user using the application may notice inconsistencies with the data and correct them. Any changes made are logged in a transformation script and when the time comes to update the data source, the transformation script is sent to the data store where the changes detailed in the script are applied.

In this project, we focus on three different ways a user can use the application to fetch data. For the rest of this paper, we shall call each way of data fetching a sub-scenario. 
\begin{enumerate}
	\item {When the user chooses to view all the columns in one table}
	\item {When the user chooses to view a subset of the columns in one table}
	\item {When the user chooses to view a subset of the columns in multiple tables}
\end{enumerate}

\noindent Our primary contributions are:
\begin{enumerate}
	\item{Defined a data transformation language using logical programming concepts to capture changes made to a relational data model}
	\item{Design decisions about how the relational model should react and handle schema changes}
	\item{Preliminary investigation into view maintenance policies for the remote datastore}
	\item{Further detailing the scenario in which an application views a subset of columns from one or more tables in a remote dataset}
	\item{Decided which issues to assign to the server side and which to the client side}
\end{enumerate}

\section{Motivation}
In our project, our main focus is on a scenario where the user views any subset of columns from one or more tables, it is important to figure out how this relates to the real world. The example in \cite{arniThesis} talks about a sports reporter who retrieves select information about basketball players to do an article on. In this case, the reporter only takes out columns regarding a player's name, his high school, and the first team he played on after entering the NBA. Compared to all the information that is stored about NBA players on a main database (e.g., salary, weight, height, number of transfers, agent name, etc.), the reporter has only taken a very small subset of columns. Another example could be a school administrator who wants to decide which students to give awards to purely based on academic merit. He/she could go to the main database storing all student information (e.g., the database could have information regarding a student's name, student number, GPA, guardian contact information, date of birth, allergy/medication information, family doctor contact, etc.) and only choose to view the student number and GPA column so that he/she will not be biased towards any names or gender.

An example where a user might need to view a subset of columns in multiple tables would be if the user is trying to find detailed information about proteins found in different microbiological organisms. The microbiological organisms and the various proteins it produces could be stored in one table while detailed information about each protein is stored in another. If a user wanted to see certain characteristics of a protein found in specific organisms, he/she would have use a subset of columns from both the microbiological organism table along with the protein time to get all the relevant information.

\subsection{Sample data used in our project}
In our project, we decided to borrow the dataset used in CPSC 304, an introductory relational database course at UBC, to use as the test dataset in our application. This was a movie star dataset that had the columns ``Star Id'', ``Name'', and ``Gender''. After supplementing some additional data to the dataset, namely the columns ``Movie Title'', ``Production Studio'', and ``Critic Rating'', we used this dataset. We also added in a second table detailing user information with the columns ``Movie Title'' which is a foreign key referencing the ``Movie Title'' column in the Movie Information table, ``Website that user information was obtained from'', ``User Rating'', and ``Number of Likes''.

\section{Solution}
The basic workflow of the scenario described in Section \ref{sec:introduction} will remain the same. However, as the user obtains more flexibility over what columns can be chosen to view in his/her application,  the complexity of the data transformations increase along with the amount of information that must be tracked (see Section \ref{sec:transformation_language}). In this project, we have decided to support the following user changes: adding a column, removing a column, editing a column name, changing a table cell's current value, removing a table cell's value to make it null, adding a row, and removing a row. Any changes that cause a violation in schema definitions (e.g., removing a row's primary key) will be resolved on the server side by the database administrator. This design decision was due to not wanting to restrict the types of users that would be able to use an app to find data to examine. As we have to assume that there could be users unfamiliar with schema/storage restrictions, we decided to ensure that schema violations are resolved by someone with the technical understanding of what error has occurred. 

\subsection{Types of Changes Supported}
Based on the various classes of changes mentioned in Wrangler \cite{kandel2011wrangler}, we decided to focus on two of them: map and reshape. Map changes deal with "[transforming] one input data row to zero, one, or multiple output rows" \cite{kandel2011wrangler}. This essentially means that the row changes that are supported in our application (e.g., add row, delete row, edit value(s) in row(s)) all fall within this class of changes. Reshape changes deal with schema changes. Any change to the columns (e.g., add/remove/rename columns) will fall into this category. The scope of this project does not deal with changing the data type of a column.

\subsection{Transformation Language}
\label{sec:transformation_language}
As we designed the transformation language, we realized it did not make sense to work off the simplest of our three sub-scenarios as the transformation language did not adapt well when the complexity of the sub-scenario was increased. Hence, we decided to work off the most complicated of our three sub-scenarios, the situation where an application has pulled a subset of columns from more than one table, and design our transformation language from that point onwards. We then retroactively checked to ensure that the transformations worked for the simpler sub-scenarios.

When designing the data transformation language, we decided to emulate some of the logical programming concepts from Prolog. We primarily included syntax concepts, ...? \textcolor{red}{don't wanna write anymore. write later} We also decided to include a timestamp at the end of each transformation to help with conflict resolution in a future stage where there may be multiple concurrent users making changes to the same set of data. Table \ref{table:transformations_summary} has a summary of how the structure of each data transformation will look like for each function.

\begin{table}[h!]
\centering
\begin{tabular}{ | c | l | }
 \hline
    User Action & Data Transformation Syntax \\ \hline \hline
    Add Row & addRow(tableName(columnName1:valuesUserPutIn1,\\ & columnName2:valuesUserPutIn2, ..., \\ & columnNameN:valuesUserPutInN),  timestamp)\\ & \\ \hline
    Remove Row & removeRow(tableName(primaryKey1:primaryKeyValue1,\\ & primaryKey2:primaryKeyValue2, ..., \\ & primaryKeyN:primaryKeyValueN),  timestamp)\\ & \\ \hline
    Add Column & addCol(tableName, nameOfNewCol, timestamp)\\ & \\ \hline
    Remove Column & removeCol(tableName, nameOfColToRemove, timestamp)\\ & \\ \hline
    Rename Column & renameCol(tableName, oldNameofCol, newNameOfCol, \\ & timestamp)\\ & \\ \hline
    Edit Value & editValue(tableName(columnNameOfEditedValue1:\\ & editedValue1, columnNameOfEditedValue2:editedValue2, ..., \\ & columnNameOfEditedValueN:editedValueN),  timestamp)\\ 
    \hline
\end{tabular}
\caption{The data transformations that will be logged to the transformation script as the user makes changes to the data.}
\label{table:transformations_summary}
\end{table}

Sections \ref{sec:add_row}, \ref{sec:remove_row},  \ref{sec:add_col}, \ref{sec:remove_col}, \ref{sec:rename_col}, and \ref{sec:edit_value} will all use the same example scenario: a journalist intern has been tasked with writing a comparison article about the different animated movies produced in the last five years and how critic and user reviews differ. She has used her application to pull data from multiple movie information tables in a remote database. The resulting set of columns that she sees in her application will be referred to as a view. In this example, the columns being viewed are ``Movie Title'', and ``Critic Rating'', which come from the Movie Information table, and the ``User Rating'', ``Number of Likes'', and ``Website that user information was obtained from'' columns, which come from the User Information table. The Movie Information table has ``Movie Title'' as the primary key while the User Information table has ``Movie Title'' as its foreign key and ``Website that user information was obtained from'' as its primary key. Table \ref{table:add_row_table_example} shows what the initial data looks like after the user has pulled from the remote database. Note that schema mappings are also pulled to the application along with the data from the remote database. The schema mappings will be used to help determine what tables each of the respective columns in the view belong to.

The following shows the table schema of the Movie Information and User Information tables:
\begin{itemize}
\item {Movie Information(Movie Title: String, Primary Key; Movie Debut Year: Int; Production Studio: String, NOT NULL; Total Gross: double; Critic Rating: String)}
\item{User Information(Movie Title: String, Foreign Key; Website that user information was obtained from: String, Primary Key; User Rating: String; Number of Likes: double)}
\end{itemize}

\begin{table}[h!]
\centering
\begin{tabular}{ | c | c | c | c | c | }
 \hline
    Movie Title & Critic  & User & Number & Website that user\\
    & Rating & Rating & of Likes & information was \\
    &&&& obtained from\\ \hline \hline
    Despicable Me 2 & 5/5 & 5/5 & 220 000 & IMDB\\ \hline
    Up & 5/5 & 5/5 & 225 000 & Facebook\\ \hline
    Kung Fu Panda 2 & 4/5 & 5/5 & 500 000 & Rotten Tomatoes\\ \hline
    Mars Needs Moms & 3/5 & 3/5 & 103 000 & Rotten Tomatoes\\
    \hline
\end{tabular}
\caption{What the application displays to the user prior to the user making any changes to the data.}
\label{table:add_row_table_example}
\end{table}

A constraint on the view is that the primary key(s) of each table have to be displayed in the view. The application must have that information as it potentially needs to have to be able to uniquely identify which value is being changed by the user (see Section \ref{sec:edit_value} for more details). Also, when a user decides to add a new row to the table (see Section \ref{sec:add_row} for more details), there needs to be values for the primary key(s) in order for a row to be added successfully. Allowing the user to choose which rows he/she wants does not guarantee that the primary key is chosen and this would mean that the user would lose certain functions without knowing why. Hence, when pulling information from each table, the application is required to pull all primary and foreign keys of the table. While we recognize that this will cause a lot of activity on the network and possibly take up much more space than required on the local machine, we could not think of a better solution. Due to this design decision, the example shown in Table \ref{table:add_row_table_example} does not include any details about whether the columns shown are the primary keys of their respective tables as we expect that the primary keys are already shown. There has been some discussion upon how to make it easier for the user to figure out which columns are primary keys and which columns come from which table but we did not feel that non-technical users could understand why certain columns were coloured the way they were. For users that can understand why columns were certain colours, it would certainly make it easier for them to perform actions such as add or remove a row (see Sections \ref{sec:add_row} and \ref{sec:remove_row}) but for the users who cannot figure out why the colours existed, it would only make it more distracting and confusing. Also, as we are working with the situation where there could possibly be third party apps accessing the database, we can only restrict what information gets pulled, we cannot really expect anything in terms of how the information is visualized.

\subsubsection{Adding a row}
\label{sec:add_row}
When the user uses his/her application to add a row in the view, it is unclear about whether or not the user desires to include an additional row in all tables that have columns in the view. One of the design decisions we have decided on is to include a new row in the tables that have a value filled in at the column. 

In Table \ref{table:add_row_table_example2}, the user has just added in a new row and populated the ``Movie Title'' and ``Critic Rating'' columns with a value. If the user decides to not populate the row any further and push his/her changes, the remote database would have the Movie Information table add a new row with Megamind as the ``Movie Title'' value and 3/5 as the ``Critic Rating'' value. As the Movie Information table has other schema restrictions, namely the restriction that the ``Production Studio'' column has to have a value, there will be an error generated when the user chooses to push to the remote database. We have decided to let the database administrator for the remote database deal with issues like these. The alternative solution is to impose another restriction upon the application such that it will have to also pull columns that have schema specifications like UNIQUE or NOT NULL. However, we felt that it would be too confusing to the user to have such columns included. With primary/foreign keys, the information captured in those columns tend to be information that give the rest of the information in the row context and meaning. Columns that have schema specifications may not have such context and meaning and thus would be confusing to impose upon the user in their application.

In Table \ref{table:add_row_table_example3}, the user has added more information about the movie ``Megamind''. As columns from the Movie Information table and the User Information table have been filled with values, there will be a new row added to the Movie Information and User Information table. Any issues with adding to the Movie Information table will have to be resolved by the local database administrator. If the user decides to try and add this row without a value for the ``Website that user has obtained rating from'' column, a primary key for the User Information table, the application will prompt the user to enter the value prior to allowing a database push. We did not feel that this was a strong or particularly confusing change to force the user to make and it would help reduce the number of issues pushed upon the database administrator.

\begin{table}[h!]
\centering
\begin{tabular}{ | c | c | c | c | c | }
 \hline
    Movie Title & Critic  & User & Number & Website that user\\
    & Rating & Rating & of Likes & information was \\
    &&&& obtained from\\ \hline \hline
    Despicable Me 2 & 5/5 & 5/5 & 220 000 & IMDB\\ \hline
    Up & 5/5 & 5/5 & 225 000 & Facebook\\ \hline
    Kung Fu Panda 2 & 4/5 & 5/5 & 500 000 & Rotten Tomatoes\\ \hline
    Mars Needs Moms & 3/5 & 3/5 & 103 000 & Rotten Tomatoes\\ \hline
    Megamind & 3/5 &  &  & \\
    \hline
\end{tabular}
\caption{The user is trying to add a new row to the table where only columns from the Movie Information table is filled.}
\label{table:add_row_table_example2}
\end{table}

\begin{table}[h!]
\centering
\begin{tabular}{ | c | c | c | c | c | }
 \hline
    Movie Title & Critic  & User & Number & Website that user\\
    & Rating & Rating & of Likes & information was \\
    &&&& obtained from\\ \hline \hline
    Despicable Me 2 & 5/5 & 5/5 & 220 000 & IMDB\\ \hline
    Up & 5/5 & 5/5 & 225 000 & Facebook\\ \hline
    Kung Fu Panda 2 & 4/5 & 5/5 & 500 000 & Rotten Tomatoes\\ \hline
    Mars Needs Moms & 3/5 & 3/5 & 103 000 & Rotten Tomatoes\\ \hline
    Megamind & 3/5 & & 200 000 & IMDB \\
    \hline
\end{tabular}
\caption{The user is trying to fill in a new row that she has added. Columns from the Movie Information and User Information table have values added to them.}
\label{table:add_row_table_example3}
\end{table}

When the user adds a row, the change that is logged to the transformation script has the following structure: addRow(tableName(columnName1:valuesUserPutIn1, columnName2:valuesUserPutIn2, ..., columnNameN:valuesUserPutInN), timestamp).

\subsubsection{Removing a row}
\label{sec:remove_row}
When removing a row, the user has to delete all values from columns that belong to a single table before the row can be successfully deleted in the database. This design decision was made to avoid users accidentally deleting whole rows when they just want to delete a single value from a column. We thought that this reflects how a user would normally use the system \textemdash if he/she really wanted to delete a row, they would go delete the whole row already. As we felt that this accurately reflected a normal usage situation, we did not consider this as a constraint that imposed extra annoyance upon the user.

When the user removes a row, the change that is logged to the transformation script has the following structure:
removeRow(tableName(primaryKey1:\\primaryKeyValue1, primaryKey2:primaryKeyValue2, ..., primaryKeyN:\\primaryKeyValueN), timestamp). We are just using the primary key to identify which specific row to delete and have decided to let the database system handle delete propagation (i.e., deleting the rows that other tables have which use values referencing the ones in the deleted row).

\subsubsection{Adding a column}
\label{sec:add_col}
When adding a column, there was a choice between having a column added to all tables that have columns in the current view or having the column added to one of the tables that had columns in the view. Originally we had the column added to all tables that have columns in the current view but quickly realized that it would lead to a lot of update problems along with the fact that it would take up extra memory. In the end, we decided to have the column added to the table with the most columns in the current view. If tables have an equal number of columns in the view, the table can be chosen in one of three ways:
\begin{itemize}
\item {Arbitrarily/at random}
\item{Based on the name of the table (e.g., add the column to the table with the name that is listed first when table names are sorted alphanumerically)}
\item{Based on the statistics of table usage that the database stores (e.g., the column could be added to the table that is queried more often). However, this would involve some back and forth dialog between the database and the remote application which may not be a good idea for performance or database security measures}
\end{itemize}

We decided to choose the second method to resolve any possible conflicts between table choice when adding columns. The add column change is logged as follows: addCol(tableName, nameOfNewCol, timestamp). 

\subsubsection{Removing a column}
\label{sec:remove_col}
While discussing this function, there was some concern about whether or not users should be allowed to remove a column and if they were, whether they should be allowed to remove primary key columns. In the end, we decided to only allow the removal of non-primary key columns for database security reasons. It could be argued that the user should be able to do whatever he/she wishes upon the data as that is the point of enabling users from all backgrounds to use and update data but upon weighing the pros and cons of this issue, it was decided that it was too easy for the user to cause major damage to the data source if primary key removal was allowed. If the user wishes to remove a primary key column, he/she could contact the database administrator. The remove column change would be logged into the transformation script as follows: removeCol(tableName, nameOfColToRemove, timestamp).

\subsubsection{Renaming a column}
\label{sec:rename_col}
Although the situation where a user would rename a column is uncommon, we decided to enable support for this use case as it could potentially come up. The transformation script syntax for this change is: renameCol(tableName, oldNameofCol, newNameOfCol, timestamp) where the tableName is the name of the table that the column to be renamed is located in, oldNameOfCol is the old name of the column used to identify which column to rename, and newNameOfCol is the new name of the column.

\subsubsection{Editing a value}
\label{sec:edit_value}
When editing a value, a user can make either a 1-1 or 1-0 change (i.e., a 1-1 change is when the user can change one value to another while a 1-0 change would be when a user deletes the value). As with the add row functionality (see Section \ref{sec:add_row}), if a user ends up violating schema constraints when editing a value, the database administrator will have to approve/deal with the changes. Some of the possible schema violations include changing a unique value to a non-unique one when the column specifies that it needs to have unique values or deleting a value from a column that has specified that it cannot be null. The edit a value change would be logged into the transformation script as follows: editValue(tableName(columnNameOfEditedValue1: editedValue1, columnNameOfEditedValue2:editedValue2, ..., columnNameOfEditedValueN:editedValueN).

\subsection{View Maintenance}
View maintenance is also something that factored into our considerations throughout the project. There were long discussions about whether or not we should allow the user to choose when updates would be pushed to the remote database or if the application should choose. In the end, we decided on a mixed-initiative approach where the user would push if certain types or a certain number of changes have occurred while also allowing the user to choose when to push changes. The primary concern was that schema changes (see Section \ref{sec:add_row}, \ref{sec:remove_row}, \ref{sec:add_col}, and \ref{sec:remove_col}) were important enough to push right away as the longer the wait period, the potential for concurrent users becomes greater. Value changes are considered less important; although value changes from one user may conflict with another user, it was decided to use the timestamp to resolve conflicts. The application can push to the remote data source after a certain number of value changes.

\subsection{Proof of Concept}
After designing the transformation language, we decided to implement the scenario workflow in order to show that our transformation language and design decisions could work to update a data source. The only extra step that was not included in \cite{arniThesis} was to allow for the user to choose which changes to sync to the remote data source. \textcolor{red}{add screen shots!! Kailang can write about all the cool things she did here!! :)}

\section{Related Work}
\textcolor{red}{need stuff here can probably reuse the stuff from the project update but will need more + editing}
\textbf{\textcolor{green}{Stuff in this section is ripped from the project update}}
One of the largest problems that we foresee is creating a transformation language to track all the possible data changes as well as how to propagate those changes correctly from one source to another. Hence, most of our related work focuses on systems that are similar or areas of work where we feel we can draw inspiration from how they tackled a similar problem to the one we are facing.

\subsection{Data Sharing Systems}
Data sharing situations are not new; there have been many collaborative data sharing systems (CDSS) \cite{ives2008orchestra, ives2005orchestra, karvounarakis2013collaborative}and data transformation language \cite{kandel2011wrangler, lakshmanan2001schemasql} created to try and address this issue. From CDSS systems, we believe we can examine how it handles schema mapping change propagations to inform our own design as this is exactly the same problem we are facing. 

\subsection{Transformation Languages}
While there are an abundance of data transformation tools \cite{raman2001potter, kandel2011wrangler}, these tools often focus on visualizing the data transformation rather than the transformation language used to represent the data changes behind the scenes. 

- something about how logical programming concepts popular with being used to declaratively represent data transformations
- find papers about this
- stuff

\subsection{View Maintenance}
Currently we have chosen to only allow data from a single table be viewed in our interface. However, given that this project could be extended in the future to include viewing data that is from multiple tables, we are looking ahead at literature in the view maintenance area to try and predict what challenges our interface might face \cite{ agrawal1997efficient, agrawal2009asynchronous, gupta1999materialized, zhou2007lazy}. We also thought that we might be able to get some ideas on how often to push any possible schema mapping/schema changes as it seems to be a comparable problem to how materialized views need to push its changes to a remote database \cite{agrawal1997efficient, agrawal2009asynchronous, zhou2007lazy}. 


\section{Future Work}
Future extensions of this work can look into testing the transformation language on other data formats like XML to see if it can effectively capture the changes that happen in an XML document. We can also look into how to handle situations where there are multiple concurrent users drawing upon the same data source and what policies should be adopted into order to achieve the best balance between performance and data consistency. Updating the data source too often could cause performance overhead from locking up the data source while updating too infrequently could cause inconsistent data to be propagated to many different users. The amount of effort and operational overhead involved with many people making the same changes would be undesirable. 

Another extension of the project could be investigating how the data transformations could be integrated into data provenance to help with data conflict resolutions \cite{arniThesis}. As the project currently stands, the remote application can push any type of data to the data source. If the data being pushed to the data source violates any table constraints (e.g., if a user tries to push a non-unique value into a column that has specified that it only takes unique values), the database administrator has to manually resolve that violation. It is possible to investigate whether or not provenance information can be leveraged to help make these decisions thus lowering the workload on the database administrator. A possible example of how to use the provenance information could be using the previous history of the sorts of changes have been accepted for that column or table to determine the likelihood of the current data violation being accepted. Provenance could also be used to determine the likelihood of correctness by the app or user who made the change; as information about who has changed values can be tracked, it is not inconceivable that the statistics of who has made the best or the most correct/acceptable changes can be used in some way.

This problem can also be investigated from an information visualization/HCI perspective. The best way to present information to the user to help facilitate use and understanding of how the data can be manipulated could be examined. However, as stated before, this is more on the application side of this problem which we are not specifically dealing with in the scope of this project. 

\bibliographystyle{abbrv}
\bibliography{final_paper_biblography}

\end{document}